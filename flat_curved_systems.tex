\documentclass[runningheads]{llncs}

\usepackage[numbers]{natbib}
\usepackage{url}
\usepackage[breaklinks=true,unicode=true,urlcolor = blue,colorlinks = true,citecolor = blue,linkcolor = blue]{hyperref}
\usepackage{amssymb}
\usepackage{amsmath}
\setcounter{tocdepth}{3}
\usepackage{graphicx}
\usepackage{nb}

%
% Define Authors
%
\definechangesauthor[name={Kostiantyn Yershov},color=blue]{KY}
\definechangesauthor[name={Oleksii Volkov},color=red]{OV}
% 
\graphicspath{{./figs/}}

\renewcommand\bibname{References}

\urldef{\mailsa}\path|{k.yershov@ifw-dresden.de,|
	\urldef{\mailsb}\path|o.volkov@hzdr.de}|   
\newcommand{\keywords}[1]{\par\addvspace\baselineskip
	\noindent\keywordname\enspace\ignorespaces#1}

\begin{document}
%\preprint{\textcolor[rgb]{0.00,0.50,0.75}{\texttt{Draft \gitAbbrevHash{} by \gitCommitterName{} on \gitCommitterIsoDate}}}	
	\mainmatter  % start of an individual contribution
	
	% first the title is needed
	\title{Flat curved systems}
	
	% a short form should be given in case it is too long for the running head
	\titlerunning{Flat curved systems}
	
	% the name(s) of the author(s) follow(s) next
	%
	% NB: Chinese authors should write their first names(s) in front of
	% their surnames. This ensures that the names appear correctly in
	% the running heads and the author index.
	%
	\author{Kostiantyn V. Yershov$^{1,2}$ \and Oleksii M. Volkov$^{3}$}
	%
	\authorrunning{Flat curved systems: Kostiantyn V. Yershov and Oleksii M. Volkov}
	% (feature abused for this document to repeat the title also on left hand pages)
	
	% the affiliations are given next; don't give your e-mail address
	% unless you accept that it will be published
	\institute{$^1$Bogolyubov Institute for Theoretical Physics of NAS of Ukraine,\\
		Metrologichna str. 14,  03143 Kyiv, Ukraine\\
		$^2$Leibniz-Institut f\"ur Festk\"orper- und Werkstoffforschung, IFW Dresden,\\
		 Helmholtzstra{\ss}e 20, D-01171 Dresden, Germany\\
		 $^{3}$Helmholtz-Zentrum Dresden-Rossendorf e. V. Institute of Ion Beam Physics and Materials Research,
		 Bautzner Landstra{\ss}e 400, 01328 Dresden, Germany\\
		\mailsa
		\mailsb}
	
	%
	% NB: a more complex sample for affiliations and the mapping to the
	% corresponding authors can be found in the file "llncs.dem"
	% (search for the string "\mainmatter" where a contribution starts).
	% "llncs.dem" accompanies the document class "llncs.cls".
	%
	
	\toctitle{Flat curved systems}
	\tocauthor{Kostiantyn V. Yershov and Oleksii M. Volkov}
	\maketitle
	
	
%\begin{abstract}
%	Abstract...
%	\keywords{Ferromagnets, curvilinear geometry, exchange interaction}
%\end{abstract}
	
\tableofcontents
\clearpage	
\section*{Discussions of the chapter}
\nb[id=KY]{
\begin{itemize}
	\item {Current title of chapter is ``\textit{Flat curved systems}''. Do you like this title? Maybe we need to change the title?, for example ``\textit{Flat curved one-dimensional systems}''.}
	\item {I propose to modify the ``Table of content'' in the following way: \begin{tcolorbox}[fonttitle=\sffamily\bfseries\large,title=New version of ``Table of content'']
		\begin{enumerate}
			\item Introduction
			\item Model of curved 1D systems
			\item Equilibrium states in curved FM wires and stripes 
			\item Linear dynamics in curved FM flat wires and stripes
			\item Domain walls in curved  FM flat wires and stripes \begin{enumerate}
					\item[5.1] Statics of DWs
					\item[5.2] Dynamics of DWs
				\end{enumerate}	
			\item Conclusions		
		\end{enumerate}
	\end{tcolorbox}}
\end{itemize}
}[30.07.2020]

\section{Introduction}\label{sec:intro}

\section{Model of curved 1D systems}\label{sec:model_1D}
%\nb[KY]{
%	In this section I propose to describe theoretical model of 1D curved uni- and biaxial ferromagnetic wires and stripes. Basically, to use here results of Refs.~\cite{Sheka15,Gaididei17a,Volkov19a} in order to introduce the energy in the curvilinear reference frame~(exchange and DMI energies).
%}%[02.04.2020]

The theoretical description of the evolution of the magnetization distribution in curved systems is based on the dynamics of the 3D vector order parameter,~e.g. the magnetization unit vector $\vec{m}=\vec{M}/M_s$, where $M_s$ is the saturation magnetization. The phenomenological description of magnetization statics and dynamics in the continuum limit relies on the total energy of the magnet~(for the simplicity we consider uniaxial anisotropy) 
\begin{equation}\label{eq:total_energy}
	E=\int\mathrm{d}V\left\{A\mathcal{E}_\textsc{e}+K\left[1-\left(\vec{m}\cdot\vec{e}_\textsc{a}\right)^2+\varepsilon\left(\vec{m}\cdot\vec{e}_\textsc{p}\right)^2\right]+\mathcal{E}_\textsc{dm}\right\}.
\end{equation}
Here, the first term in~\eqref{eq:total_energy} describes the isotropic exchange interaction $\mathcal{E}_\textsc{e}=\sum_{i=x,y,z}\left(\vec{\nabla} m_i\right)^2$ with with the exchange stiffness $A$. The unit vectors $\vec{e}_\textsc{a}$ and $\vec{e}_\textsc{p}$ in Eq.~\eqref{eq:total_energy} gives the direction of the easy- and hard-axis anisotropy directions, respectively; $K>0$ is the easy-axial anisotropy constant and $\varepsilon>0$ is an anisotropy ratio. Such kind of anisotropy is effectively induced by the magnetostatic interaction in the thin stripes. In the curved ferromagnetic systems magnetostatic interaction can be reduced to the easy-tangential anisotropy for wires with circular/square cross-section~\cite{Slastikov12} or biaxial anisotropy with eassy-plane perpendicular to the wires plane for rectangular cross sections~\cite{Aharoni98,Gaididei17a}, which simply results in a shift of anisotropy constants. For thin, narrow, and curved stripes (ribbons) the approximation of the shape anisotropy is used also for inhomogeneous magnetization states~\cite{Gaididei17a}, including domain walls~(DWs) in wires~\cite{Yershov15b,Yershov16,Pylypovskyi16} and stripes~\cite{Yershov18a,Volkov19c}. This approximation accurately describes dynamics of a transversal DW in thin curvilinear systems even for a magnetically soft ($K = 0$) materials~\cite{Yershov15b,Yershov16,Pylypovskyi16,Yershov18a,Volkov19c}.  The competition between exchange and anisotropy results in the magnetic length $\ell=\sqrt{A/K}$, which determines a length scale of the system.  The last term in~\eqref{eq:total_energy} represents DMI contribution  $\mathcal{E}_\textsc{dm}$.

We consider a 1D thin nanowire/stripe whose transverse size is small enough to ensure magnetization uniformity along the cross section direction. One can describe a wire using the Frenet--Serret parametrization for a 3D curve $\vec{\gamma}$. We use its natural parametrization by arc length $s$\footnote{It means that $|\partial_s\vec{\gamma}|=1$.} of general form $\vec{\gamma}=\vec{\gamma}(s)$. In Cartesian basis $\hat{\vec{x}}_i\in\{\hat{\vec{x}},\hat{\vec{y}},\hat{\vec{z}}\}$, one can parameterize the curve as $\vec{\gamma}=\gamma_i\hat{\vec{x}}_i$\footnote{The Einstein summation convention is used here and everywhere below.}. Let
us introduce the local normalized curvilinear basis~(Frenet--Serret frame):
\begin{equation}\label{eq:FrenetSerret_basis}
	\vec{e}_\textsc{t}=\partial_s \vec{\gamma},\quad \vec{e}_\textsc{n}=\partial_s\vec{e}_\textsc{t}/|\partial_s\vec{e}_\textsc{t}|,\quad \vec{e}_\textsc{b}=\vec{e}_\textsc{t}\times\vec{e}_\textsc{n}
\end{equation}
with $\vec{e}_\textsc{t}$ being the tangent, $\vec{e}_\textsc{n}$ being the normal, and $\vec{e}_\textsc{b}$ being the binormal to the curve $\vec{\gamma}$. The differential properties of the curve are determined by the Frenet--Serret formulae:
\begin{equation}\notag%\label{eq:FrenetSerret_formulae}
	\partial_s\vec{e}_\alpha = \mathcal{F}_{\alpha\beta}\vec{e}_\beta,\quad \|\mathcal{F}_{\alpha\beta}\|=\left(
		\begin{matrix}
			0&\kappa&0\\
			-\kappa&0&\tau\\
			0&-\tau&0
		\end{matrix}\right),
\end{equation}
where $\kappa$ and $\tau$ are curvature and torsion of the curve $\vec{\gamma}$. Here, Greek indices $\{\alpha,\, \beta\} = \{\textsc{t},\, \textsc{n},\, \textsc{b}\}$ numerate the curvilinear coordinates and the curvilinear components of vector fields.
Using Frenet--Serret reference frame we can introduce the parametrization of physical wire/stripe with a finite cross section size as
\begin{equation}\notag
\vec{r}\left(s,\zeta_1,\zeta_1\right)=\vec{\gamma}+\zeta_1\vec{e}_\textsc{n}+\zeta_2\vec{e}_\textsc{b},\quad \sqrt{\zeta_1^2+\zeta_2^2}\leq \ell,
\end{equation}
where $\zeta_i$ are coordinates within the cross section. The assumption of the magnetization one-dimensionality can be formalized as $\vec{m}=\vec{m}(s)$.

In the curvilinear reference frame~\eqref{eq:FrenetSerret_basis} the exchange energy density $\mathcal{E}_\textsc{x}$ reads as~\cite{Sheka15}
\begin{equation}\label{eq:exhange_FrenetSerret}
\begin{split}
\mathcal{E}_\textsc{x}=\mathcal{E}_\textsc{x}^0+\mathcal{E}_\textsc{x}^\textsc{a}+\mathcal{E}_\textsc{x}^\textsc{dm},\quad &\mathcal{E}_\textsc{x}^0=|\partial_s\vec{m}|,\\
\mathcal{E}_\textsc{x}^\textsc{a}=\mathcal{K}_{\alpha\beta}m_\alpha m_\beta,\quad\mathcal{E}_\textsc{x}^\textsc{dm}=\mathcal{F}_{\alpha\beta}&\left(m_\alpha \partial_s m_\beta-m_\beta\partial_s m_\alpha\right).
\end{split}
\end{equation}
Here, the first term $\mathcal{E}_\textsc{x}^0$ describes the common isotropic part of the exchange expression, i.e. formally it has the same form as for the straight wire/stripe. The second term in~\eqref{eq:exhange_FrenetSerret} $\mathcal{E}_\textsc{x}^\textsc{a}$ determines the effective geometry-induced anisotropy-like contribution with anisotropy constant $\mathcal{K}_{\alpha\beta}=\mathcal{F}_{\alpha\gamma}\mathcal{F}_{\beta\gamma}$. The last term $\mathcal{E}_\textsc{x}^\textsc{dm}$ coincides with a combination of Lifshitz invariants, therefore can be interpreted as an effective geometry-induced DMI with DMI strength $\mathcal{F}_{\alpha\beta}$ linear with respect to curvature and torsion, which allows strong chiral effects in curvilinear 1D FMs. Exchange-driven DMI constant were experimentally observed in the parabolas geometry~\cite{Volkov19c} with values up to $\approx0.65$~mJ/m$^2$~(see Fig.~\ref{fig:dmi_parabola}), which is comparable with experimentally reported values obtained for asymmetric Co sandwiches~\cite{Boulle16}.

%==================================================================\
\begin{figure}[t]
	\includegraphics[width=\textwidth]{fig_dmi_parabola}
	\caption{\label{fig:dmi_parabola}%
		\textbf{Geometry-induced DMI constant.} Dependencies of the geometry-induced DMI constants $D^\textsc{e}$ on curvature $\kappa_0$ for different parabolic stripe widths. The inset
			pictures represent the distribution of the $D^\textsc{e}$ along the parabolic stripe apex. Scale bars correspond to 100 nm for all the inset pictures. Figure is reproduced from~\cite{Volkov19c}.}
\end{figure}
%==================================================================/

The DMI energy density for the curvilinear 1D system reads as~\cite{Volkov18}
\begin{equation}\label{eq:DMI_FrenetSerret}
\begin{split}
\mathcal{E}_\textsc{dm}=\mathcal{E}_\textsc{dm}^0+\mathcal{E}_\textsc{dm}^\textsc{a},\quad \mathcal{E}_\textsc{dm}^0=\mathfrak{D}_{\alpha\beta}\left(m_\alpha \partial_s m_\beta-m_\beta\partial_s m_\alpha\right),\ \quad \mathcal{E}_\textsc{dm}^\textsc{a}=\mathfrak{K}_{\alpha\beta}\,m_\alpha m_\beta,\\
\mathfrak{D}_{\alpha\beta}=\frac{1}{2}\left(\begin{matrix}
0 & -D_\textsc{b} & -D_\textsc{n}\\
D_\textsc{b} & 0 & -D_\textsc{t}\\
D_\textsc{n} & D_\textsc{t} & 0
\end{matrix}\right),\quad \mathfrak{K}_{\alpha\beta}=\frac{1}{2}\left(\begin{matrix}
2\tau D_\textsc{t} & \tau D_\textsc{n} & \kappa D_\textsc{t} + \tau D_\textsc{b}\\
\tau D_\textsc{n} & 0 & \kappa D_\textsc{n}\\
\kappa D_\textsc{t} + \tau D_\textsc{b} & \kappa D_\textsc{n} & 2\kappa D_\textsc{b}
\end{matrix}\right).
\end{split}
\end{equation}
Here, $\vec{D}=\left(D_\textsc{t},D_\textsc{n},D_\textsc{b}\right)$ is a DMI vector with $D_\alpha$ being the curvilinear components. The first term $\mathcal{E}_\textsc{dm}^0$ in \eqref{eq:DMI_FrenetSerret} describes the full set of Lifshitz invariant in the Frenet--Serret reference frame, while the second term $\mathcal{E}_\textsc{dm}^\textsc{a}$ determines the effective geometry-induced anisotropy with anisotropy tensor $\mathfrak{K}_{\alpha\beta}$.

The total energy~\eqref{eq:total_energy} in the curvilinear reference frame reads as
\begin{equation}\label{eq:totalEnergy_FrenetSerret}
\frac{E}{K\mathcal{S}\ell}=\int\limits_{-\infty}^{+\infty}\left[m'_\alpha m'_\alpha+\mathcal{D}_{\alpha\beta}\left(m_\alpha m'_\beta - m'_\alpha m_\beta\right)+\mathcal{K}^{\gamma}_{\alpha\beta}m_\alpha m_\beta\right]\mathrm{d}\xi.
\end{equation}
Here, $\mathcal{S}$ is a cross section area, prime denotes to the derivative with respect to the dimensionless coordinate $\xi=s/\ell$. The DMI $\mathcal{D}_{\alpha\beta}=\ell\mathcal{F}_{\alpha\beta}+\mathfrak{D}_{\alpha\beta}/\sqrt{AK}$ and anisotropy $\mathcal{K}^\gamma_{\alpha\beta}=\ell^2\mathcal{K}_{\alpha\beta}+\mathfrak{K}_{\alpha\beta}/\sqrt{AK}-\delta_{\gamma\alpha}\delta_{\gamma\beta}$ matrices are defined as
\begin{equation}
\begin{split}
\mathcal{D}_{\alpha\beta}=&\left(\begin{matrix}
	0&\varkappa-d_\textsc{b}/2&-d_\textsc{n}/2\\
	-\varkappa+d_\textsc{b}/2&0&\sigma-d_\textsc{t}/2\\
	d_\textsc{n}/2&-\sigma+d_\textsc{t}/2&0
\end{matrix}\right),\\
\mathcal{K}_{\alpha\beta}^\gamma=&\frac{1}{2}\left(\begin{matrix}
2\varkappa^2+2\sigma d_\textsc{t}&\sigma d_\textsc{n}&-2\varkappa\sigma+\varkappa d_\textsc{t}+\sigma d_\textsc{b}\\
\sigma d_\textsc{n}&2\varkappa^2+2\sigma^2&\varkappa d_\textsc{n}\\
-2\varkappa\sigma+\varkappa d_\textsc{t}+\sigma d_\textsc{b}&\varkappa d_\textsc{n}&2\sigma^2+\varkappa d_\textsc{b}
\end{matrix}\right)-\delta_{\gamma\alpha}\delta_{\gamma\beta},
\end{split}
\end{equation}
where $\varkappa=\ell \kappa$ and $\sigma=\ell\tau$ are dimensionless curvature and torsion, respectively, and $d_\alpha=D_\alpha/\sqrt{AK}$ is a dimensionless components of DMI vector. One should note that we did not introduce any new interactions in the model \eqref{eq:total_energy} we only present it in the curvilinear reference frame.
\section{Geometry effects in statics and dynamics in curved systems}\label{sec:effects_1D}

\subsection{Equilibrium states in FM curved wires and stripes}\label{sec:statics}
%\nb[KY]{
%	I propose to discuss here the statics in the curvilinear 1D system. I think we can discus here:
%	\begin{itemize}
%		\item equilibrium states in rings~\cite{Sheka15,Gaididei17a,Klaui03a}
%		\item equlibrium states in curved wires + DWs~\cite{Nahrwold09,Lewis09,Kim14,Yershov15b,Gaididei18a,Volkov19,Volkov19c,Korniienko19b}
%	\end{itemize} 
%}%[02.04.2020]

The developed approach in Refs.~\cite{Sheka15,Volkov18} and described in Sec.~\ref{sec:model_1D} enables us to obtain a general static solution for the high anisotropy case. We consider a physically interesting case of easy-tangential anisotropy, which favours the magnetization distribution tangential to the wire. In the strong anisotropy limit, the magnetization is quasitangential; therefore we expect small deviation of magnetization from tangential direction, i.e. we introduce a small deviation\footnote{We use the following parametrization for the unit magnetization vector $\vec{m}=\sin\theta\cos\phi\,\vec{e}_\textsc{t}+\sin\theta\sin\phi\,\vec{e}_\textsc{n}+\cos\theta\,\vec{e}_\textsc{b}$.} as $\theta=\pi/2+\vartheta$ and $\phi=\phi_0+\varphi$, whereas $|\vartheta|,\ |\varphi|\ll1$ and $\cos\phi_0=\pm1$. Then the total energy density~\eqref{eq:totalEnergy_FrenetSerret} for $\vec{d}=\vec{0}$ can be rewritten as follows $\mathcal{E} \approx \mathcal{E}_\textsc{t}-2\left(\varkappa\varphi'-\cos\phi_0\, \varkappa\sigma\,\vartheta\right)+\left(\vartheta^2+\varphi^2\right)+\text{const}$. Here the first term~$\mathcal{E}_\textsc{t}$ is the energy density of a strictly tangential distribution, the second term is a sum of geometry-induced DMI and anisotropy terms, and the last term determines the anisotropy contribution. Minimization of the energy with respect to $\vartheta$ and $\varphi$ results in equilibrium state
\begin{equation}\label{eq:strong_anis_lomit}
\theta=\frac{\pi}{2}-\varkappa\sigma\cos\phi_0,\quad \phi=\phi_0+\varkappa'.
\end{equation}
Here and bellow we consider planar systems with $\sigma=0$. In this case, Eq.~\eqref{eq:strong_anis_lomit} shows that magnetization of the equilibrium state
always lies within the plane of the wire~(i.e. $\vec{m}\cdot\vec{e}_\textsc{b}=0$), and it is not tangential to the wire if $\varkappa'\neq0$. For the case of ring-shaped wire with $\varkappa=\text{const}$ one obtains a well known \textit{vortex} magnetization distribution~\cite{Klaui03a,Sheka15,Guimaraes17}. This state is typical for relatively large rings~(rings with small curvature) and describes the homogeneous magnetization distribution in the curvilinear reference of frame, see Fig.~\ref{fig:ring_states}. In the opposite case of small rings one obtains a quasi-uniform in Cartesian reference of frame magnetization distribution, see Fig.~\ref{fig:ring_states}. This state is an arrangement of spins in which the ring is divided into two magnetic domains separated by two DW, with magnetizations oriented tangentially in two different directions, clockwise and counterclockwise, a structure that is usually referred to as an \textit{onion} state~\cite{Klaui03a}.

%==================================================================\
\begin{figure}[t]
	\includegraphics[width=\textwidth]{fig_ring_states}
	\caption{\label{fig:ring_states}%
		\textbf{Equilibrium states in nanorings.} (a) Magnetization distribution of the ground state in a ring wire with different reduced curvatures: vortex state for $\varkappa<\varkappa_c$ and onion state for $\varkappa>\varkappa_c$. (b) Hysteresis curve of polycrystalline  Co nanorings array exhibiting vortex and onion states. Red arrows indicate the field path used to obtain the rings in the states imaged with PEEM in (c). (c) A PEEM image of	four polycrystalline Co rings. (d) Spin structure of (d$^1$) a vortex wall and (d$^2$) a transverse wall simulated using the OOMMF code. PEEM images of (d$^3$) a 30 nm thick and 530 nm wide, (d$^4$) a 10 nm thick and 260 nm wide, and	(d$^5$) a 3 nm thick and 730 nm wide ring. The gray scale shows	the magnetization direction. Panel (a) is reproduced from~\cite{Sheka15}; (b) and (c) are reproduced from~\cite{Klaui03a}; (d) is reproduced from~\cite{Laufenberg06}.}
\end{figure}
%==================================================================/

Nanorings with \textit{vortex} magnetization ground state can be prepared with smaller radii than nanodisks with the same spin structure~\cite{Kravchuk07}. This nanoring critical radius which separates the \textit{vortex} and \textit{onion} states defined, in general, by the relation~\cite{Sheka15}:
\begin{equation}\notag
R_{c} = \varkappa_c^{-1}\sqrt{A/K},\quad \varkappa_c \approx 0.657.
\end{equation}
For the permalloy nanoring we have $R_{c}\approx 17$ nm, i.e. for rings with $R>R_c$ one obtains \textit{vortex} state, while for $R<R_c$ -- \textit{onion} state.

As it was mentioned above that two domains in the \textit{onion} state are separated by two DWs. Depending on the width of the ring one can obtain a transverse head-to-head~(tail-to-tail) DWs for narrow ribbons, or vortex head-to-head~(tail-to-tail) DWs for wide rings, see Fig.~\ref{fig:ring_states}.

%==================================================================\
\begin{figure}[t]
	\includegraphics[width=\textwidth]{fig_cornu_n_meander}
	\caption{\label{fig:cornu_n_meander}%
		\textbf{Equilibrium states in curved wires.} (a) and (c) Spatial distribution of magnetization (red arrows) along the wires with the shape of meander and Euler spiral; the inclination angle $\phi$ is shown in insets. (b) Inclination angle $\phi$ as function of arc length for meander shaped wire with $\varkappa_0=0.6$: line -- analytical solution, dots -- results of numerical simulations. Panels (a) and (b) are reproduced from~\cite{Korniienko19b}.}
\end{figure}
%==================================================================/

The deviation of magnetization from  strictly tangential direction in circle segment with $\varkappa<\varkappa_c$ can be observed in wires with periodically repeated semicircles of curvature $\varkappa_0$~\cite{Korniienko19b}, see Fig.~\ref{fig:cornu_n_meander}(a). The spatial distribution of the curvature of such a wire is the square-wave function $\varkappa(\xi) = (-1)^{\lfloor\xi/\xi_0\rfloor}\varkappa_0$ with period $2\xi_0$, $\xi_0=\pi/\varkappa_0$, and $\lfloor x \rfloor$ defines the integer part of $x$. The corresponding inclination $\phi$ for square-wave function $\varkappa(\xi)$ reads~\cite{Korniienko19b}
\begin{equation}\label{eq:phi_meander}
\phi = (-1)^\lambda \text{am}\left[\frac{\xi-\xi_0\left(\lambda+1/2\right)}{k},ik\right],\quad \lambda=\lfloor \xi/\xi_0 \rfloor,\quad k=\frac{1}{\sqrt{\varkappa_0^2-\sin^2\varphi_0}},
\end{equation}
where $\text{am}\left(x,y\right)$ is Jacobi amplitude~\cite{NIST10}. Constant $\varphi_0=|\phi\left(n\xi_0\right)|$ is maximal value of the function $\phi(\xi)$ [see Fig.~\ref{fig:cornu_n_meander}(b)], it is determined by the equation $2kF(\varphi_0,ik) = \xi_0$, where $F(x,y)$ is elliptic integral of the first kind~\cite{NIST10} and the modulus $k = k(\varphi_0)$ is defined in~\eqref{eq:phi_meander}. One should note that the curvature amplitude $\varkappa_0$ is the only parameter which controls the system. The maximal deviation $\varphi_0$ from the tangential direction takes place in points of junction of two semicircles, this is because of the curvature jump. Depending on the $\varkappa_0$ one obtains two different behaviors~\cite{Korniienko19b}: (i) In the limit case of small curvature $\varkappa_0\ll1$ and due to the easy-tangential anisotropy the wire is magnetized practically tangentially except the junction points, where the magnetization demonstrates the small deviations of amplitude $\varphi_0\approx\varkappa_0$. (ii) In the opposite case of large curvature $\varkappa_0\gg1$ exchange interaction is dominating and results in quasi uniform magnetization aligned with $x$-axis with $\varphi_0\lesssim\pi/2$.

For the specific case of the wire with constant gradient of the curvature $\varkappa'=\chi=\text{const}$ [wire with a shape of Euler spiral~\cite{Lawrence14}, also known as Cornu spiral or clothoid, see Fig.~\ref{fig:cornu_n_meander}(c)], geometry-induced magnetization inclination from the tangential direction can be calculated analytically~\cite{Yershov18a}. One can find that magnetization deviates from the tangential direction as $\phi=\phi_0+\arcsin(2\chi)/2$~\cite{Yershov18a}, also see Fig.~\ref{fig:cornu_n_meander}(c). This solution coincides with~\eqref{eq:strong_anis_lomit} for the limit case of small gradient of the curvature $\chi\ll1$. 

%For the case of non-zero gradient of the curvature, geometry-induced magnetization inclination from the tangential direction can be calculated analytically~\cite{Yershov18a}. For the specific case of the wire with constant gradient of the curvature $\varkappa'=\chi=\text{const}$ [wire with a shape of Euler spiral~\cite{Lawrence14}, also known as Cornu spiral or clothoid] one can find that magnetization deviates from the tangential direction as $\phi=\phi_0+\arcsin(2\chi)/2$~\cite{Yershov18a}, also see Fig.~\ref{fig:1d_states}(a). This solution coincides with~\eqref{eq:strong_anis_lomit} for the limit case of small gradient of the curvature $\chi\ll1$. 

Similar geometry-induced inclination of magnetization from anisotropy axis were also obtained for wires with box-function curvature~\cite{Gaididei18a}. 


\subsection{Linear dynamics in FM curved flat wires and stripes}\label{sec:linear_dyn}
\nb[KY]{
	In this section we can discuss the spin-wave dynamics in curved systems:
	\begin{itemize}
		\item magnon spectrum in the ring~\cite{Sheka15}
		\item local modes \& magnon filter ~\cite{Gaididei18a,Korniienko19b}
	\end{itemize}
}%[02.04.2020]

\subsection{Domain walls in FM curved flat wires and stripes}\label{sec:DW_dyn}
%\nb[KY]{
%	In this section we can discuss the dynamics of transversal/vortex DWs in the curved systems, namely rings~\cite{Richter16,Mawass17,Moreno17a,Garg17}, wires with non-zero gradient of the curvature~\cite{Yershov15b,Yershov18a,Volkov19c}.
%}%[02.04.2020]

\begin{tcolorbox}[fonttitle=\sffamily\bfseries\large,title=Some introduction...]
	Novel concepts for magnetic logic~\cite{Allwood05} and storage~\cite{Parkin08} devices rely on DW motion inside curvilinear segments, e.g. U-shaped segments in the magnetic racetrack memory~\cite{Parkin08}. Therefore, the modification of dynamic and static properties of a DW due to the wire curvature is of high importance for the applications.
\end{tcolorbox}

\subsubsection{Statics of DW.} To analyze the statics and dynamics of a DW in a curved magnetic stripe we use a collective variable approach based on the $q-\Phi$ model~\cite{Slonczewski72,Malozemoff79}
\begin{equation}\label{eq:q_phi_model}
\cos\theta=-p\tanh\left[\frac{\xi-q(t)}{\Delta}\right],\quad\phi=\Phi(t),
\end{equation}
where $\{q, \Phi\}$ are time-dependent conjugated collective variables, which determine the DW position and orientation of transverse magnetization component, respectively; $\Delta$ is a DW width; $p$ is a topological charge, which determines the DW type: head-to-head ($p = +1$) or tail-to-tail ($p = −1$). Here $q$ and $\Delta$ are dimensionless quantities measured in the units of magnetic length $\ell$. The Ansatz~\eqref{eq:q_phi_model} coincides with the exact DW solution for a rectilinear wire ($\varkappa=0$ and $\vec{d}=\vec{0}$). Here, the curvature is considered as a small perturbation, which keeping the form~\eqref{eq:q_phi_model} unchanged. The DW width $\Delta$ is shown to be a slaved variable~\cite{Hillebrands06} even in the flat curvilinear systems~\cite{Yershov18a}, i.e. $\Delta(t) = \Delta\left[q(t),\Phi(t)\right]$.

The energy~\eqref{eq:totalEnergy_FrenetSerret} for DW reads
\begin{equation}\label{eq:dw_energy}
	\frac{E^\textsc{dw}}{2K\mathcal{S}\ell}\approx\frac{1}{\Delta}+\Delta\,\left(1+\varepsilon\sin^2\Phi\right)+p\pi\varkappa(q)\cos\Phi+\mathcal{O}\left(\varkappa^2\right),
\end{equation}
where the condition $\varkappa\Delta\ll1$ was applied when integrating. Similar expression were obtained for the DW energy in the circular wire segment~\cite{Kruger07a}. The first and second terms represent the competition between the isotropic exchange and anisotropy terms, and defines the DW width in rectilinear system. While the last terms determine the geometry-induced DMI and anisotropy terms, respectively.

Firstly, lest us consider the systems with localized curvature $\varkappa(\pm\infty)=0$, i.e. parabola-shaped system. Minimization of~\eqref{eq:dw_energy} with respect to $q$ and $\Phi$ results in the following equilibrium values
\begin{equation}\label{eq:dw_eq_state}
\varkappa'\left(q_0\right)=0,\quad\cos\Phi_0=-p.
\end{equation}
The first equation in~\eqref{eq:dw_eq_state} corresponds to DW pinning at the maximum of the curvature. While, the second equation demonstrates process of a symmetry breaking due to the geometry-induced DMI~\cite{Yershov15b}: transverse magnetization of a head-to-head (tail-to-tail) DW is directed outwards (inwards), see Fig.~\ref{fig:dw_wire_1}(a). The latter effect was recently observed experimentally~\cite{Kim14,Volkov19c}, see Figs.~\ref{fig:dw_wire_1}(b),~\ref{fig:dw_wire_1}(d),~\ref{fig:dw_wire_1}(e). There is an intuitive explanation: the choice of $\Phi_0$ defined by~\eqref{eq:dw_eq_state} makes the magnetization distribution more homogeneous, see Figs.~\ref{fig:dw_wire_1}(a) and~\ref{fig:dw_wire_1}(b), which decreases energy of the geometry-induced DMI term in~\eqref{eq:dw_energy}. One should note that such a phase selectivity for DWs also takes place in the systems with constant curvature~\cite{Moreno17a} and in rings in the onion state, see Figs.~\ref{fig:ring_states}(a) and \ref{fig:ring_states}(d$^2$). It should be noted that circular geometry does not produce any geometrical pinning potential due to the constant curvature $\varkappa=\text{const}$, which results in undefined $q_0$. Nevertheless, the domain wall can have an equilibrium position in circular segment in the presence of an external magnetic field~\cite{Kruger07a,Jamali11}.

%==================================================================\
\begin{figure}[t]
	\includegraphics[width=\textwidth]{fig_dw_curved_wire}
	\caption{\label{fig:dw_wire_1}%
		\textbf{DW in a flat wire bend.} (a) Equilibrium state of the transverse DW at the wire bend obtained by means of micromagnetic simulations. (b) Magnetic states imaged by XMCD-PEEM of tail-to-tail and head-to-head DWs. (c) Scanning electron microscopy image of two curved planar nanowires embedding a strip line (yellow shaded), which generates out-of-plane magnetic	fields with opposite direction at each nanowire. (b) and (c) show the scanning transmission x-ray microscopy images visualizing DWs. (d) Initial and final displacement of DWs after five subsequent field pulses as indicated in (a). In (d) the upper DW moves a distance of $840\pm20$nm from position (1) to position (3), while the lower DW travels $940\pm20$nm from position (2) to position (4). Scale bar, 1 mm. (e) Same for upper nanowire and opposite current pulses. DW travels ($680 \pm 20$) nm to the left and ($340\pm20$) nm to the right, respectively. Scale bar: 500 nm. Panel (a) is reproduced from~\cite{Yershov15b}; panel (b) is reproduced from~\cite{Volkov19c}; panels (c), (d), and (e) are reproduced from~\cite{Kim14}.}
\end{figure}
%==================================================================/

\subsubsection{Dynamics of DW.} The dynamics of magnetization is governed by the phenomenological Landau--Lifshitz--Gilbert equation. In terms of $\{q,\Phi,\Delta\}$ it reads~\cite{Yershov15b,Yershov18a}
\begin{equation}\label{eq:dw_qPhi_motion}
\begin{split}
\frac{\alpha}{\Delta}\dot{q}+p\dot{\Phi} = &-p\pi\frac{\partial\varkappa(q)}{\partial q}\cos\Phi,\\
p\dot{q}-\alpha\Delta\dot{\Phi} = &-p\pi\varkappa(q)\sin\Phi+\varepsilon\Delta\sin 2\Phi,
\end{split}
\end{equation}
where $\alpha$ is a Gilbert damping parameter, overdot correspond to the derivative with respect to dimensionless time $\overline{t}=t\gamma_0K/M_s$ with $\gamma_0$ being the gyromagnetic ratio and $M_s$ saturation magnetization.
The dynamics of the DW width can be described as $\alpha\pi^2\dot{\Delta}/12=1/\Delta-\Delta\left(1+\varepsilon\sin^2\Phi\right)$, which shows that $\Delta$ relaxes towards its equilibrium value $\Delta_0=1/\sqrt{1+\varepsilon\sin^2\Phi}$. The characteristic time of this relaxation is proportional to the damping $\propto \alpha$~\cite{Hillebrands06}. Usually $\alpha\ll1$, therefore one can conclude that the DW width is a slave variable $\Delta(t) = \Delta[\Phi(t)]$ and DW dynamics can be described by the set~\eqref{eq:dw_qPhi_motion}  with the equilibrium DW width $\Delta=\Delta_0$. From \eqref{eq:dw_qPhi_motion} it follows that the gradient of the curvature is a driving force for DWs. The physical origin of this force is the geometry-induced DMI driven by the exchange.

%==================================================================\
\begin{figure}[t]
	\includegraphics[width=\textwidth]{fig_dw_dynamics_curved}
	\caption{\label{fig:dw_wire_2}%
		\textbf{DW dynamics in curved wires.} (a) Eigenfrequency of the DW oscillations in vicinity of the equilibrium -- the extreme point of parabolic wire bend. (b) DW velocity $\dot{q}$ as a function of the gradient of the curvature in the Euler spiral. In (a) and (b)  lines correspond to the analytical predictions and symbols to the results of micromagnetic simulations. Panel (a) is reproduced from~\cite{Yershov15b}; panel (b) is reproduced from~\cite{Yershov18a}.}
\end{figure}
%==================================================================/

The equations of motion~\eqref{eq:dw_qPhi_motion} can be linearized with respect of small deviations of DW position $\tilde{q}=q-q_0$ and phase $\tilde{\phi}=\Phi-\Phi_0$ in the vicinity of equilibrium position~\eqref{eq:dw_eq_state}. For the limit case of small curvature $\varkappa\ll1$ and $\varepsilon=0$ the equations of motion~\eqref{eq:dw_qPhi_motion} linearized with respect to the deviations reads~\cite{Yershov15b}
\begin{equation}\label{eq:dw_qPhi_motion_linear}
\left(1+\alpha^2\right)\left\|\begin{matrix}
\dot{\tilde{q}}\\
\dot{\tilde{\phi}}
\end{matrix}\right\|=\pi\left\|\begin{matrix}
0 & \varkappa(q_0)\\
\varkappa''(q_0)&-\alpha\varkappa(q_0)
\end{matrix}\right\|\cdot\left\|\begin{matrix}
\tilde{q}\\
\tilde{\phi}
\end{matrix}\right\|.
\end{equation}
For the case of small damping the solution of~\eqref{eq:dw_qPhi_motion_linear} results in harmonic decaying oscillations, see Fig.~\ref{fig:dw_wire_1}(d), with frequency $\Omega$ and modified effective friction $\eta$ defined as
\begin{equation}
\Omega \approx \pi\sqrt{|\varkappa(q_0)\varkappa''(q_0)|},\quad \eta\approx \alpha\frac{\pi}{2}\varkappa(q_0).
\end{equation}
Eigenfrequency of the DW oscillations in vicinity of the equilibrium position is presented in Fig.~\ref{fig:dw_wire_2}(a). As it was mentioned above, the circular geometry does not generate any pinning potential, therefore the eigenfrequency of the DW oscillations in such geometry is $\Omega=0$. The value of pinning potential increases with the increasing of the curvature, see Fig.~\ref{fig:dw_wire_2}(a), which results in the increasing of the deppining field up to 60-80 mT~\cite{Volkov19c,Lewis09}.

Remarkably, that for the wires with monotonic increasing/decreasing curvature as a function of the arc length the curvature gradient in Eqs.~\eqref{eq:dw_qPhi_motion} results in a driving force, i.e. DW can be moved without any external driving~(field or current) and pinning. In the limit case of Euler spiral the DW can move with constant velocity
\begin{equation}\label{eq:velocity_euler_spiral}
V=-p\,\mathcal{C}\,\frac{\varkappa'}{\alpha},\quad\cos\Phi_0=\mathcal{C}=\pm1.
\end{equation}
The phase during the dynamics behaves as $\Phi\approx\Phi_0+p/\left(\alpha V \overline{t}\right)$, which in long time approximation results in $\Phi\left(\overline{t}\to\infty\right)=\Phi_0$. Hence, the parameter $\mathcal{C}$ is interpreted as the DW magnetochirality~\cite{Kim14}.  The corresponding geometry-induced DW velocity for the Euler spiral reaches up to 150 m/s, see Fig.~\ref{fig:dw_wire_2}(b). For the case of curved wires~\cite{Lewis09,Nahrwold09,Wartelle18} with curvature in the range $\kappa\in\left[0;1/150\right]$ nm$^{-1}$ the corresponding curvature-induced DW velocity is expected to be about 80 m/s.

The direction of the geometry-induced DW motion\footnote{One has to notice a one-to-one correspondence between natural	parameter $\xi$ and gradient of the curvature $\varkappa'$ with DW magnitochirality $\mathcal{C}$: change of the natural parameter sign results in the changing of gradient of the curvature and DW magnitochirality	signs, therefore direction of DW motion physically is the same.} is determined by the product of the DW magnetochirality, topological charge, and gradient of the curvature $V\propto p\,\mathcal{C}\varkappa'$, see Fig.~\ref{fig:dw_wire_2}(b). The geometry-induced dynamics are accompanied by the DW motion in the area of bigger curvature. In this way, in the energy~\eqref{eq:dw_energy} the geometry-induced DMI term becomes dominant. This term fixes the DW phase $\cos\Phi=\pm1$, which depends on the sign of the product of the topological charge $p$ and curvature $\varkappa$, see Eq.~\eqref{eq:dw_energy}. Therefore, the transition to the precessional regime becomes suppressed. This effect can be interpreted as the curvature-induced suppression of the Walker limit~\cite{Yershov18a}. This is in contrast to the case of field-driven DWs in a straight wire, where the phase of the DW is not fixed (Zeeman term in the energy of DW is independent of the phase~\cite{Malozemoff79,Hillebrands06}).

One can see that DW velocity is increasing with decreasing Gilbert damping~$\alpha$, see Eq.~\eqref{eq:velocity_euler_spiral}. For the limit case of vanishing damping~($\alpha\to0$) the velocity of the DW increases exponentially with time, and its transverse magnetization component orients perpendicular to the wires plane with $\Phi(\alpha=0)\to-\mathcal{C}\pi/2$. 

The DW velocity~\eqref{eq:velocity_euler_spiral} is similar to the well-known expression $V_u = u \beta/\alpha$ in magnetic stripes with biaxial anisotropy caused by the Zhang--Li mechanism~\cite{Bazaliy98,Zhang04}, where $\beta$ is a nonadiabatic spin-transfer parameter. Current-induced translational DW motion takes place only if $u<u_\textsc{w}$, where $u_\textsc{w}$ is Walker current~\cite{Thiaville05}. However, for the case of a geometry-induced motion, a Walker-limit-like effect of the transition to the precessional regime does not appear and the DW demonstrates a high-speed translational motion without any external driving. However, for the current-induced dynamics of DW in an uniaxial~($\varepsilon=0$) circular-shaped wires~($\varkappa=\text{const}$) curvature results in the appearance of Walker limit~\cite{Yershov16}
\begin{equation}
u_\textsc{w}^\varkappa=\frac{\alpha}{|\alpha-\beta|}\pi\varkappa.
\end{equation}

Geometry-induced motion of DWs desctibed by Eqs.~\eqref{eq:dw_qPhi_motion} can be compared with so called automotion of DW in curved systems~\cite{Chauleau10,Richter16,Mawass17}. This type of motion can be realized relying on: (i) The transformation of DWs from transverse to vortex types after the action of current pulse~\cite{Chauleau10}, see Figs.~\ref{fig:automotion}(a)-\ref{fig:automotion}(f). In this case, the motion of DW is caused by the modification DW phase $\Phi$~(a canonically conjugated variable to DW position $q$). Such type of motion is possible for the duration of excitation is short compared to the relaxation time of the DW structure~\cite{Chauleau10}. (ii) The second mechanism is relying on the motion of DWs in systems with coordinate-dependent cross sectional area of a nanostripe~\cite{Richter16,Mawass17}, see Fig.~\ref{fig:automotion}(g). In this case, the motion of DW is caused by the minimization of DW energy in the narrow area of asymmetric ring~\cite{Mawass17}. One should note that with increasing the gradient of ring's aspect ratio the velocity of DW increases.

%==================================================================\
\begin{figure}[t]
	\includegraphics[width=\textwidth]{fig_automotion}
	\caption{\label{fig:automotion}%
		\textbf{Automotion of DWs.} (a)-(b) Magnetic force microscopy images of the initial and final magnetic states of the entire S-shaped nanostrip showing two transverse and two vortex DWs, respectivelly . In (b) two vortex DWs displaced by 1.5 and 1.7 $\mu$m after transforming under a 1 ns current pulse of 3.6 TA/m$^2$ amplitude. (c) and (d) zoom on the vortex DWs, with schematics shown in (e) and (f). (g) Time-resolved STXM XMCD snapshots of automotive DWs motion during the switching process from the onion to the vortex state in the ferromagnetic ring. Red and blue lines in (g) illustrate the averaged trajectory of the vortex DWs. Panels (a)-(f) are reproduced from~\cite{Chauleau10}; panel (g) is reproduced from~\cite{Mawass17}.}
\end{figure}
%==================================================================/

\subsection{Geometry effects in statics and dynamics of Skyrmions in curved 2D systems}\label{sec:effects_2D}
\nb[KY]{
	This subsection is \textbf{questionable}. Should we consider geometry effects in \textit{flat} curved stripes discussed  in Refs.~\cite{Zhang18e,Liu20,Luis19}?
}%[02.04.2020]

\section{Conclusions}

\bibliographystyle{splncs}
\bibliography{soliton}
\end{document}
