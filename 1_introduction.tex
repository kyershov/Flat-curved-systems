\section*{Introduction}\label{sec:intro}

The investigation of an interplay between geometry and topology of the order parameter has become one of the key research fields in the modern soft and solid state physics, as it provides insight on the modification of material responses by geometrical tailoring. This result provides means both for fundamental study and technological applications: the presence of curvature in a generic electronic system leads to the appearance of scalar and vector geometric potentials, that introduce anisotropic and chiral responses to the system, respectively. This allows to fabricate novel logical and signal elements, that will meet growing technological demands for the high-density energy efficient electronics. For instance, novel fundamental responses were observed in thin layers of superconductors~\cite{Tempere09,Ying17}, superfluids~\cite{Kuratsuji12}, nematic liquid crystals~\cite{Lopez-Leon11}, cell membranes~\cite{McMahon05}, semiconductors~\cite{Gentile15,Ortix15} and magnetism~\cite{Streubel16}. 

Recent advances in the fabrication of curvilinear magnets \cite{Streubel16a,Fernandez17,Ball17,Sanz-Hernandez18,Huth18,Huth20,Sanz-Hernandez20} raises the question of the influence of curvature on statics and dynamics of a magnetization. For spintronic devices the geometrically induced potentials arising from curved shapes gain practical importance. For example, designs of these devices require wires with curvilinear segments, e.g. vertical U-shaped configuration increases the storage density of the potential magnetic racetrack memory devices~\cite{Parkin08,Zhang15}.

The purpose of this section is to introduce the general theory for the description of the flat curved one-dimensional systems. The described methods are valid for systems with non-zero local curvature~(curved wires and narrow ribbons), and can not be used for description of boundary effects, e.g.  stabilization of vortex and antivortex in magnetically soft disk~\cite{Shinjo00} and astroid~\cite{Shigeto02} geometries, respectively.

The paper is organized as follows. 