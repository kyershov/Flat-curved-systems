\documentclass[showpacs,amsmath,amssymb,aps,pra,longbibliography,
10pt,preprint,superscriptaddress,showkeys]{revtex4-1}
\usepackage{graphicx}
\usepackage[breaklinks=true,colorlinks=true,linkcolor=blue,urlcolor=blue,citecolor=blue]{hyperref}
\usepackage{graphics}
\usepackage{dcolumn}
\usepackage{bm,mathtools,amsmath,amssymb,amsbsy,mathdots}
\usepackage{natbib}
\usepackage{soul,color}
\usepackage{epstopdf}
\usepackage{todonotes}
\usepackage[note-name]{notes2bib}
%\linespread{1.8}

\newcommand{\Yershov}[1]{\todo[inline,caption={},color=green!10]{\textbf{\underline{Kostiantyn Yershov:}}\\#1}}
\newcommand{\Volkov}[1]{\todo[inline,caption={},color=blue!10]{\textbf{\underline{Oleksii Volkov:}}\\#1}}

\begin{document}

	\title{Flat curved systems}

	\date{\today}
	
	\author{Kostiantyn~Yershov}
	\email{k.yershov@ifw-dresden.de}
	\affiliation{Bogolyubov Institute for Theoretical Physics of NAS of Ukraine,\\
		Metrologichna str. 14,  03143 Kyiv, Ukraine}
	\affiliation{Leibniz-Institut f\"ur Festk\"orper- und Werkstoffforschung,\\
		IFW Dresden,\\
		Helmholtzstra{\ss}e 20, D-01171 Dresden, Germany}
	
	\author{Oleksii~Volkov}
	\email{o.volkov@hzdr.de}
	\affiliation{Helmholtz-Zentrum Dresden-Rossendorf e.~V. \\
		Institute of Ion Beam Physics and Materials Research, \\
		Bautzner Landstra{\ss}e 400, 01328 Dresden, Germany}
	
%	\maketitle
	
\begin{abstract}
	Curvature effects in magnetism offer appealing possibilities to obtain new magnetic textures and fundamental effects at the nanoscale due to the interplay between the magnetic geometry and topology of the order parameter. Here we review main theoretical and experimental curvature-driven effects in flat magnetic systems.  
	\keywords{Ferromagnets, curvilinear geometry, exchange interaction}
\end{abstract}
	
\maketitle
\tableofcontents
\clearpage	
\section{Introduction}\label{sec:intro}

The interplay between geometry and topology is one of the fundamental properties in soft and condensed matter physics, including thin layers of superconductors~\cite{Tempere09,Ying17}, superfluids~\cite{Kuratsuji12}, nematic liquid crystals~\cite{Lopez-Leon11}, cell membranes~\cite{McMahon05}, semiconductors~\cite{Gentile15,Ortix15} and magnetism~\cite{Streubel16}. This provides instruments to introduce novel properties to the conventional magnetic systems by tailoring their geometry: the theoretical investigations in this area provide fundamental insight into the behavior of curved magnetic nanostructures and the control of their excitations, while the experimental investigations together with novel fabrication methods pave the way to the realization of artificial chiral nanostructures with the predefined properties and responses. Thus, the emergent field of curvilinear magnetism defines both the fundamental background to predict and experimental applicabilities to observe the curvature-induced magnetochiral effects and tolopologically-induced magnetic textures. 



\section{Theoretical model of a curved 1D system}\label{sec:model_1D}
\Yershov{(02.04.2020) In this section I propose to describe theoretical model of 1D curved uni- and biaxial ferromagnetic wires and stripes. Basically, to use here results of Refs.~\cite{Sheka15,Gaididei17a,Volkov19a} in order to introduce the energy in the curvilinear reference frame~(exchange and DMI energies).}

The theoretical description of the magnetization distribution in the framework of continuum model is represented by a three-dimensional (3D) unit vector field  $\vec{m}=\vec{M}/M_s$ with $M_s$ being the saturation magnetization. Despite that the history of the curvature-induced effects in vector-field model is rather long~\cite{Kamien02,Bowick09,Turner10}, the majority of theoretical approaches were limited by the vector field being tangential to the curvilinear geometry. A fully 3D approach was developed recently for thin magnetic shells~\cite{Gaididei14} and wires~\cite{Sheka15} of arbitrary shape and it is based on the assumption that the magnetic anisotropy suppress the dipolar interaction in a curvilinear magnet. Thus, in this case, the total micromagnetic energy will have the form 
\begin{equation}\label{eq:total_energy_1}
E=\int\mathrm{d}V\left[\mathcal{E}_\textsc{ex}+\mathcal{E}_\textsc{am}\right],
\end{equation}
where $\mathcal{E}_\textsc{ex} = A \, \sum_{i=x,y,z}\left(\vec{\nabla} m_i\right)\cdot \left(\vec{\nabla} m_i\right)$ is the energy density of the isotropic exchange interaction with $A$ being the exchange stiffness, and $\mathcal{E}_\textsc{am}$ is the density of the effective anisotropy interaction. It is important to note that for the case of curvilinear magnetic object, the sample geometry appears implicitly in \eqref{eq:total_energy_1} only through the anisotropy term due to the spatial variation of the anisotropy axis. 

%For instance, in the prototypical case of a 1D uniaxial curved magnet $\mathcal{E}_\textsc{am} = K\left(\vec{m}\cdot\vec{e}_\textsc{a}\right)^2$, where the unit vector $\vec{e}_\textsc{a}$ determines the direction of the anisotropy axis. 

To remove the coordinate dependence of the magnetic anisotropy term $\mathcal{E}_\textsc{am}$, it is instructive to make a transition to the local curvilinear frame of reference, where the spatial variation of the anisotropy axes will be automatically accounted for, and the anisotropy energy density assumes its usual translation-invariant form. To show the far-reaching consequences of such transition, let us illustrate the prototypical model of the 1D wire, which is described by a curve $\vec{\gamma}(s)$ with a fixed cross-section area $S$ and parameterized by arc length $s\in\left[0,L\right]$, where $L$ is the length of the wire. To make the transition to the local curvilinear frame of reference, it is convenient to use the Frenet-Serret equations,  
\begin{equation}\label{eq:FrenetSerret_basis}
\vec{e}_\textsc{t}=\partial_s \vec{\gamma},\qquad \vec{e}_\textsc{n}=\frac{\partial_s\vec{e}_\textsc{t}}{|\partial_s\vec{e}_\textsc{t}|},\qquad \vec{e}_\textsc{b}=\vec{e}_\textsc{t}\times\vec{e}_\textsc{n}
\end{equation}
with $\vec{e}_\textsc{t}$, $\vec{e}_\textsc{n}$ and $\vec{e}_\textsc{b}$ being the tangent, the normal and the binormal vectors to the curve $\vec{\gamma}$, respectively. The differential properties of the curve are determined by the Frenet--Serret formulae:
\begin{equation}\notag%\label{eq:FrenetSerret_formulae}
\partial_s\vec{e}_\alpha = \mathcal{F}_{\alpha\beta}\vec{e}_\beta, \qquad
 \|\mathcal{F}_{\alpha\beta}\|=
\begin{Vmatrix*}  0 \, \, \,	& \kappa(s) & \, \, \, 0 \, \, \, \\ -\kappa(s) \, \, \, & 0  & \, \, \, \tau(s) \, \, \, \\  0 \, \, \, & -\tau(s) & \, \, \, 0 \, \, \,
\end{Vmatrix*}, \qquad \{\alpha,\, \beta\} = \{\textsc{t},\, \textsc{n},\, \textsc{b}\},
\end{equation}
where $\kappa(s)$ and $\tau(s)$ are curvature and torsion of the curve $\vec{\gamma}(s)$. Here, Greek indices numerate the curvilinear coordinates and the curvilinear components of vector fields. 

The transition to the local curvilinear frame of reference~\eqref{eq:FrenetSerret_basis} leads to the restructuring in all magnetic energy terms containing spatial derivatives. A characteristic example is the transformation of the exchange energy density term $\mathcal{E}_\textsc{ex}$, which acquires tow additional terms, namely the effective curvature-induced chiral $\mathcal{E}_\textsc{ex}^\textsc{dmi}$ and anisotropy $\mathcal{E}_\textsc{ex}^\textsc{an}$ terms~\cite{Sheka15,Gaididei17a}:
\begin{equation}\label{eq:exhange_FrenetSerret}
\begin{split}
&\mathcal{E}_\textsc{ex}=\mathcal{E}_\textsc{ex}^0+\mathcal{E}_\textsc{ex}^\textsc{dm}+\mathcal{E}_\textsc{ex}^\textsc{an}, \\
\mathcal{E}_\textsc{ex}^0=A\,|\partial_s\vec{m}|^2, \quad \quad \mathcal{E}_\textsc{ex}^\textsc{an}=&A\,\mathcal{K}_{\alpha\beta}m_\alpha m_\beta, \quad \quad \mathcal{E}_\textsc{ex}^\textsc{dmi}=A\,\mathcal{F}_{\alpha\beta}\left(m_\alpha \partial_s m_\beta-m_\beta\partial_s m_\alpha\right), \\
 \|\mathcal{K}_{\alpha\beta}\| =& \begin{Vmatrix*}  \kappa(s)^2 \, \, \,	& 0 & \, \, \, - \kappa(s) \tau(s) \, \,  \\ 0 \, \, \, & \kappa(s)^2+\tau(s)^2  & \, \, \, 0 \, \, \\  - \kappa(s) \tau(s) \, \, \, & 0 & \, \, \, \tau(s)^2 \, \, 
 \end{Vmatrix*},
\end{split}
\end{equation}
where the Einstein notation is used for summation. Here, \textit{the first term} $\mathcal{E}_\textsc{ex}^0$ describes the common isotropic part of the exchange expression, i.e. formally it has the same form as for the straight wire or stripe. \textit{The second term} $\mathcal{E}_\textsc{ex}^\textsc{an}$ represents curvature-induced anisotropy and have significant effect in the ground-state magnetization profile, e.g. tilt of the equilibrium magnetization state even in the case of curved nanowire with strong uniaxial anisotropy~\cite{Sheka15,Pylypovskyi16}. The curvature-induced anisotropy coefficients are proportional to the second degree of local curvature $\kappa(s)$ and torsion $\tau(s)$. \textit{The third term} $\mathcal{E}_\textsc{ex}^\textsc{dmi}$ in~\eqref{eq:exhange_FrenetSerret} contains the set Lifshitz invariants in the Frenet--Serret frame of reference, which indicates its chiral properties and formal similarity to Dzyaloshinskii-Moriya interaction (DMI)~\cite{Sheka15}. It is worth noting, that this chiral term is purely determined by the sample geometry, namely, local curvature and torsion, while the conventional spin-orbit driven DMI originates in bulk magnetic crystals with low symmetry~\cite{Dzyaloshinsky58,Moriya60a} or at interfaces between a ferromagnetic and nonmagnetic material with strong spin-orbit coupling~\cite{Fert90,Crepieux98,Bode07,Yang15}. Thus, for the case of simplicity and generality, in the following we will refer the curvature-induced DMI as the \textit{extrinsic} to the crystal or layer stack (\textit{e}DMI), while the spin-orbit driven one will be referred as \textit{intrinsic} DMI (\textit{i}DMI), due to its nature. By its symmetry the \textit{e}DMI vector is always perpendicular to the beding plane of a curvilinear wire~\cite{Sheka15}: $D^\textsc{E} = -2 A \, \tau(s) \, \vec{e}_\textsc{t} - 2 A \, \kappa(s) \, \vec{e}_\textsc{b}$~\cite{Pylypovskyi16,Volkov18}.

It should be noted that all these curvature-related effects appear in systems with geometry broken symmetry without the introduction of any new interactions to the model~\eqref{eq:total_energy_1}, but by a mathematically rigorous restructuring of conventional micromagnetic terms in the local curvilinear frame of reference~\eqref{eq:FrenetSerret_basis}. The chiral and anisotropy terms are sources of emergent ``vector'' and ``scalar'' potentials, respectively for spin-waves in a curved wire~\cite{Sheka15}, which is similar to geometrical potentials in curvilinear quantum-mechanics systems~\cite{Ortix15}.

%This term reveals itself in the domain wall (DW) pinning at localized wire~\cite{Yershov15b} and stripe~\cite{Volkov19c} bends and it is responsible for the existence if magnetochiral effects in curvilinear magnetic systems~\cite{Hertel13a}, e.g. negative DW mobility for helical wires~\cite{Sheka15,Yershov16,Pylypovskyi16}.


\section{Geometry effects in statics and dynamics in flat curved systems}\label{sec:effects_1D}
In the case of flat curved systems $\tau(s)=0$, that will be discussed in this chapter, the resulting \textit{e}DMI vector aligned along the $\hat{z}$ axis: $D^\textsc{E} = - 2 A \, \kappa(s) \, \hat{z}$, which favors a specific direction of the in-plane magnetization rotation. It is insightful to investigate flat curvilinear systems with distinct curvature distributions: 
\begin{itemize}
	\item Constant curvature;
	\item Curvilinear distribution with local gradient;
	\item System with periodically changing curvature;
	\item Curvature distribution with non-zero curvature gradient along the wire or stripe. 
\end{itemize}

\subsection{Nanowire ring}\label{subsec:nanowire_ring}
The prototypical example of the applicability of our approach is the nanowire ring with $\kappa(s)=\kappa=\textrm{const}$ and easy tangential anisotropy $K<0$. In this case the curve distribution is following $\vec{\gamma}(s)=\left\lbrace \kappa^{-1} \cos(\kappa s), \kappa^{-1} \sin(\kappa s), 0 \right\rbrace$. The total energy \eqref{eq:total_energy_1} of the wire with area cross-section $S$ has the form
\begin{equation} \label{eq:total_energy_ring_1}
	E = S \int_{0}^{2 \pi/\kappa} \left\lbrace A \left[ |\partial_s\vec{m}|^2 + 2 \, \kappa \, (m_\textsc{t} \partial_s{m}_\textsc{n}-m_\textsc{n} \partial_s{m}_\textsc{t}) + \kappa^2 (m^2_\textsc{t}+m^2_\textsc{n}) \right] - K \, m^2_\textsc{t} \right\rbrace.
\end{equation}
For the sake of simplicity, it is useful to parameterize the unit magnetization vector as follows
\begin{equation} \label{eq:angular_parameterization_1}
	\vec{m}=\sin\theta \, \sin \phi \, \vec{e}_\textsc{t} + \sin\theta \, \cos \phi \, \vec{e}_\textsc{n} + \cos \theta \, \vec{e}_\textsc{b},
\end{equation}
where angular variables $\theta$ and $\phi$ depend on spatial and temporal coordinates. Using this parameterization \eqref{eq:angular_parameterization_1} and introducing the angular variable $\chi=\kappa s$ with the reduced curvature $\varkappa=\kappa \ell_\textrm{m}$, the total energy \eqref{eq:total_energy_ring_1} will be simplified to the following form~\cite{Sheka15}
\begin{equation} \label{eq:total_energy_ring_2}
	E = \dfrac{S \sqrt{A K}}{\varkappa} \int_{0}^{2 \pi} \left\lbrace \varkappa^2 \left[ (\partial_\chi \theta)^2 + \sin^2 \theta \, (1- \partial_\chi \phi)^2 \right] - \sin^2\theta \, \sin^2 \phi \right\rbrace.
\end{equation}
The minimization of the energy \eqref{eq:total_energy_ring_2} results in $\theta=\pi/2$ and the azimuthal angle $\phi$, which satisfies the pendulum equation
\begin{equation} \label{eq:pendulum}
	\varkappa^2 \, \partial_{\chi\chi} \phi + \sin \phi \, \cos \phi = 0.
\end{equation}
The homogeneous solution of it in the curvilinear frame of references corresponds to the planar vortex state:
\begin{equation} \label{eq:vortex_solution}
	\phi^\textrm{vor} = C \dfrac{\pi}{2}, \qquad \qquad \theta^\textrm{vor} = \dfrac{\pi}{2},
\end{equation}
which is well known for the magnetic nanorings~\cite{Klaui03a,Guimaraes17} and in the following will be referred as a \textit{vortex} solution. The parameter $C=\pm1$ is called the vortex chirality and it defines the direction of magnetization rotation along the ring: clockwise $C=+1$ and counterclockwise $C=-1$. The resulting total energy of the vortex state is the following
\begin{equation} \label{eq:vortex_energy}
	E^\textrm{vor} = \dfrac{S \sqrt{A K}}{\varkappa} \left( \varkappa^2 - 1 \right).
\end{equation}

%==================================================================\
\begin{figure}
	{\centering
		\begin{tikzpicture}%[scale=1,show grid={left,above,true}]
		\node[above] at (-0.5,0.7) {\includegraphics[width=0.23\textwidth]{../figs/figure_1a}};
		\node[above] at (4.5,0.25) {\includegraphics[width=0.23\textwidth]{../figs/figure_1b}};
		\node[above] at (9.5,0.25) {\includegraphics[width=0.23\textwidth]{../figs/figure_1c}};
		\node[above] at (-0.5,-0.25) {\textbf{(a)} $\varkappa <\varkappa _c$};
		\node[above] at (4.5,-0.25) {\textbf{(b)} $\varkappa =0.7$};
		\node[above] at (9.5,-0.25) {\textbf{(c)} $\varkappa =2$};
		\end{tikzpicture}
	}
	\caption{Magnetisation distribution of the ground state in a ring wire with different reduced curvatures $\varkappa $: \textbf{(a)} vortex state~\eqref{eq:vortex_solution}, \textbf{(b)} and \textbf{(c)} onion states~\eqref{eq:onion_solution}.}
	\label{fig:ring}
\end{figure}
%==================================================================/

An inhomogeneous solution of the pendulum equation \eqref{eq:pendulum} reads 
\begin{equation} \label{eq:onion_solution}
	\phi^\textrm{on} = \textrm{am}(x,k), \qquad \qquad x=\dfrac{2 \chi}{\pi} \textrm{K}(k), \qquad \qquad \theta^\textrm{on} = \dfrac{\pi}{2},
\end{equation} 
where $\textrm{am}(x, k)$ is Jacobi's amplitude~\cite{NIST10} and the modulus $k$ is determined by the condition 
\begin{equation}
	2 x \, k \, \textrm{K}(k) = \pi,
\end{equation}
with $\textrm{K}(k)$ being the complete elliptic integral of the first kind~\cite{NIST10}. The corresponding magnetization solution is analogous to a well-known onion state~\cite{Klaui03a,Guimaraes17} typical for the ring geometry. In the following we will refer to \eqref{eq:onion_solution} as the \textit{onion} state. The total energy of this state reads
\begin{equation} \label{eq:onion_energy}
	E^\textrm{on} = \dfrac{S \sqrt{A K}}{\varkappa} \left[ \dfrac{4 \varkappa}{\pi \, k} \textrm{E}(k) - \varkappa^2 - \dfrac{1}{k^2} \right],
\end{equation}  
where $\textrm{E}(k)$ is the complete elliptic integral of the second kind~\cite{NIST10}. The equality of energies for the vortex \eqref{eq:vortex_energy} and onion \eqref{eq:onion_energy} solutions $E^\textrm{vor}=E^\textrm{on}$ determines th critical reduced curvature $\varkappa_c\approx0.657$, which separates the vortex state ($\varkappa< \varkappa_c$) and the onion one ($\varkappa>\varkappa_c$). The typical magnetization distribution is shown in Fig.~\ref{fig:ring}.

To analyze the magnons distribution in the nanowire ring system, we linearize the Landau-Lifshitz equation \eqref{eq:Landau-Lifshitz_1} on the background of $\theta_0 = \pi/2$ and $\phi_0$, where $\phi_0$ corresponds either to the vortex state $\phi_0 = \phi^\textrm{vor}$ or to the onion state $\phi_0 = \phi^\textrm{on}$, depending on the reduced curvature of the system $\varkappa$. For the small deviations $\vartheta = \theta-\theta_0$ and $\varphi=\phi-\phi_0$, one could obtain the set of linear equations:
\begin{subequations} \label{eq:linear_set-eqs}
	\begin{equation} \label{eq:linear_set}
	%\fl
	\left[-\varkappa ^2\partial_{\chi \chi } + V_1(\chi )\right]\vartheta  = -\partial_\tau  \varphi ,\quad \left[-\varkappa ^2\partial_{\chi \chi } + V_2(\chi )\right]\varphi  = \partial_\tau  \vartheta ,
	\end{equation}
where $\partial_\tau $ is the derivative with respect to the dimensionless time $\tau =\Omega _0 t$ with $\Omega _0=2\omega _0|\lambda |$. Here the ``potentials'' $V_1(\chi )$ and $V_2(\chi )$ are as follows:
	\begin{equation} \label{eq:potentials}
	%\fl
	V_1(\chi ) = \sin^2\Phi _0 - \varkappa ^2\left[1-\partial_\chi  \Phi _0(\chi )\right]^2,\qquad V_2(\chi ) = - \cos 2\Phi _0(\chi ).
	\end{equation}
\end{subequations}

To solve the set of linear equations~\eqref{eq:linear_set} we apply the partial wave expansion
\begin{equation} \label{eq:partial-wave}
	\begin{split}
		&\vartheta(\chi, \tau) = \sum_{m=0}^{\infty} \, \vartheta_m \, \cos \left( m \, \chi - \Omega \, \tau + \delta_m \right), \\
		&\varphi(\chi, \tau) = \sum_{m=0}^{\infty} \, \varphi_m \, \cos \left( m \, \chi - \Omega \, \tau + \delta_m \right),
	\end{split}
\end{equation} 
where $m$ is the azimuthal quantum numbers, $\delta_m$ is arbitrary phases and $\Omega=\omega/\Omega_0$ is the dimensionless frequency. Let us mention that Eqs.~\eqref{eq:linear_set} for the partial waves $\vartheta _m$ and $\varphi _m$ are invariant under the conjugation $\Omega \to-\Omega $, $m\to-m$, $\delta _m\to-\delta _m$, $\vartheta _m\to\vartheta _m$, and $\varphi _m\to-\varphi _m$. In classical theory we can choose any sign of frequency; nevertheless, to make a contact with a quantum mechanics with a positive frequency and energy $\mathscr{E}_k=\hslash \omega _k$, we discuss the case $\Omega >0$ only.

First we consider the magnons on the background of the vortex state \eqref{eq:vortex_solution}. In this case $V_1=1-\varkappa ^2$ and $V_2=1$. By substituting the expansion \eqref{eq:partial-wave} into Eqs.~\eqref{eq:linear_set-eqs} one can calculate the following spectrum of magnon eigenstates:
\begin{equation} \label{eq:spectrum-vortex}
\Omega _m^{\mathrm{vor}}(\varkappa ) = \sqrt{\left( 1+ \varkappa^2 \, m^2\right)\left( 1+ \varkappa^2 \, m^2-\varkappa ^2\right)}.
\end{equation}
The lower eigenfrequencies are plotted in the Fig.~\ref{fig:circle_spectrum}.

In the limit case of a quasi-straight wire ($\varkappa \to0$) the magnon frequencies read
\begin{equation*}
\Omega _m^{\mathrm{vor}}(\varkappa ) = 1-\frac{\varkappa ^2}{2}+\varkappa ^2 \, m^2+ O\left( \varkappa^4 \right).
\end{equation*}
Thus the curvature decreases the gap as compared to the case of the straight wire $(\varkappa =0)$  with dispersion $\Omega _s(\mathfrak{K}) = 1+ \mathfrak{K}^2$, where $\mathfrak{K}=\varkappa \, m$ is the corresponding normalised wave vector. 

%==================================================================\
\begin{figure}
	{\centering
		\includegraphics[width=0.55\columnwidth]{../figs/figure_2}
	}
	\caption{The lowest eigenfrequencies of linear excitations in a ring nanowire depending on the curvature $\varkappa$.}
	\label{fig:circle_spectrum}
\end{figure}
%==================================================================/

Let us consider now the magnons on background of the onion state \eqref{eq:onion_solution}. By substituting $\Phi _0=\Phi ^\mathrm{on}$ into \eqref{eq:potentials} one can present the potentials $V_1(\chi )$ and $V_2(\chi )$ as the following Fourier expansions \cite{NIST10}
\begin{equation} \label{eq:V12-series}
	\begin{split}
		&\qquad \qquad \qquad \qquad V_1=A_0+\sum\limits_{n=1}^\infty A_n\cos(2n\chi ), \\
		&A_0=\frac{1}{k^2}-\frac{4}{\pi }\frac{\varkappa }{k}\mathrm{E}(k)+\varkappa ^2, \qquad \qquad \qquad A_n=8\varkappa ^2q^n\left[\frac{1}{1+q^{2n}}-\frac{2n}{1-q^{2n}}\right],\\ 
		&\qquad \qquad \qquad \qquad V_2=B_0+\sum\limits_{n=1}^\infty B_n\cos(2n\chi ),\\
		&B_0=\frac{2}{k^2}-\frac{4}{\pi }\frac{\varkappa }{k}\mathrm{E}(k)-1, \qquad \qquad \qquad \, \, B_n=-16\varkappa ^2\frac{nq^n}{1-q^{2n}}.
	\end{split}
\end{equation}
where Jacobi's nome $q$ is given in terms of the modulus $k$ by $q=\exp\left(-\pi \mathrm{K}(\sqrt{1-k^2})/\mathrm{K}(k)\right)$ \cite{NIST10}. At critical point $\varkappa _c$ the nome $q(\varkappa _c)\approx 0.135$, and its value rapidly tends to zero with $\varkappa $. Thus we can restrict ourselves with few lower Fourier harmonics.

Now by substituting \eqref{eq:partial-wave} and \eqref{eq:V12-series} into \eqref{eq:linear_set} and multiplying the Fourier series, we get the following set of equations
\begin{equation} \label{eq:phi-theta-onion}
	\begin{split}
		&(\varkappa^2 \, m^2+A_0)\vartheta _m+\frac12\sum\limits_{n=1}^\infty A_n(\vartheta _{m+2n}+\vartheta _{m-2n}) = \Omega \, \varphi _m,\\
		&(\varkappa^2 \, m^2+B_0)\varphi _m+\frac12\sum\limits_{n=1}^\infty B_n(\varphi _{m+2n}+\varphi _{m-2n}) = \Omega \, \vartheta _m,	
	\end{split}
\end{equation}
where the conventional rule $f_{-|n|}=f_{|n|}$ is used for the amplitudes $\vartheta _{n}$ and $\varphi _{n}$.

We do not possess the exact solution of the infinite set of equations \eqref{eq:phi-theta-onion}. As a first approach, by neglecting the modes coupling one obtains
\begin{equation} \label{eq:Omega-no-coupl}
\Omega _m^{(u)} = \sqrt{(\varkappa ^2m^2+A_0)(\varkappa ^2m^2+B_0)}.
\end{equation}
The coupling results in the mixing of different partial waves. However the influence of coupling decreases with $n$ due to the rapid decay of $A_n$ and $B_n$, hence \eqref{eq:Omega-no-coupl} provides good enough estimation of frequencies for not very small azimuthal quantum number $m$.

An exception is $\Omega =0$: in this case the zero (Goldstone) mode is realised due to arbitrary direction of the onion axis. This eigenstate has the following form
\begin{equation} \label{eq:phi-Goldstone}
\varphi ^{\mathrm{G}}(\chi ) = \partial_\chi  \Phi ^{\mathrm{on}}(\chi ) \propto \mathrm{dn}(x,k),\qquad \qquad \vartheta ^{\mathrm{G}}(\chi )=0, \qquad \qquad \Omega ^{\mathrm{G}}=0.
\end{equation}
Using the Fourier expansion of Jacobi's function $\mathrm{dn}(x,k)$, one can easily see that the Goldstone mode $\varphi ^{\mathrm{G}}(\chi )$ contains infinite number of partial waves, hence the coupling between different partial waves for this mode is crucial. One has to stress that as distinct from the vortex case, eigenstates on the background of the onion state do not coincide with partial waves: each eigenstate with eigenfrequency $\Omega _n$ corresponds to a set of partial waves with different azimuthal quantum numbers $m$ due to the coupling. The lowest eigenfrequencies, calculated using \eqref{eq:phi-theta-onion} with account of only four lowest partial waves $\Omega _n$, are plotted in the Fig.~\ref{fig:circle_spectrum}.

The spectrum of the narrow nanorings is well studied experimentally \cite{Giesen07,Demokritov09}. It should be noted that the typical for experiments ring radii $R$ are about hundreds of nanometres, while the typical magnetic length $w$ is about ten nanometres, hence the dimensionless curvature $\varkappa \approx w/R\ll1$. That is why in most of experiments the ground state of the ring is the vortex state and the onion one appears only under influence of external magnetic field \cite{Klaui03a}.

One has to stress that we do not discuss here the influence of the dipolar interaction on the magnetisation structure supposing that the thickness is much smaller than the exchange length. Nevertheless it is instructive to compare our results for the critical curvature $\varkappa _c$ with the boundary between different phases in magnetic rings. Our case of circumference--wire corresponds to the very narrow ring. It is well known \cite{Kravchuk07} that depending on the geometrical and magnetic parameters of the nanoring, there exist different magnetic phases in magnetically soft ring: easy-axis, easy-plane, and planar vortex phases. The lowest bound for the vortex state magnetic ring is given by the triple point $R^{\mathrm{(tr)}}\approx \ell \sqrt{3}$ for the infinitesimally narrow ring  \cite{Kravchuk07}. For rough estimation of the critical curvature we can simply replace the magnetic length $w$ by the exchange length $\ell$, hence $\varkappa \approx \ell/R^{\mathrm{(tr)}} = 1/\sqrt{3}\approx0.577$, which is close to $\varkappa _c\approx 0.657$. One has to note that the monodomain state in \cite{Kravchuk07} was supposed to be the easy-axial one instead of the onion state.


\subsection{Equilibrium states in FM curved wires and stripes}\label{sec:statics}
\Yershov{(02.04.2020) I propose to discuss here the statics in the curvilinear 1D system. I think we can discus here: 
\begin{itemize}
	\item equilibrium states in rings~\cite{Sheka15,Gaididei17a,Klaui03a}
	\item equlibrium states in curved wires + DWs~\cite{Nahrwold09,Lewis09,Kim14,Yershov15b,Gaididei18a,Volkov19,Volkov19c,Korniienko19b}
\end{itemize} 
}


\subsection{Linear dynamics in FM curved flat wires and stripes}\label{sec:linear_dyn}
\Yershov{(02.04.2020) In this section we can discuss the spin-wave dynamics in curved systems:
	\begin{itemize}
		\item magnon spectrum in the ring~\cite{Sheka15}
		\item local modes \& magnon filter ~\cite{Gaididei18a,Korniienko19b}
	\end{itemize}
}

\subsection{Dynamics of DWs in FM curved flat wires and stripes}\label{sec:DW_dyn}
\Yershov{(02.04.2020) In this section we can discuss the dynamics of transversal/vortex DWs in the curved systems, namely rings~\cite{Richter16,Mawass17,Moreno17a,Garg17}, wires with non-zero gradient of the curvature~\cite{Yershov15b,Yershov18a,Volkov19c}.        n this section we can discuss the dynamics of transversal/vortex DWs in the curved systems, namely rings~\cite{Richter16,Mawass17,Moreno17a,Garg17}, wires with non-zero gradient of the curvature~\cite{Yershov15b,Yershov18a,Volkov19c}.
}


\subsection{Geometry effects in statics and dynamics of Skyrmions in curved 2D systems}\label{sec:effects_2D}
\Yershov{(02.04.2020) This subsection is \textbf{questionable}. Should we consider geometry effects in \textit{flat} curved stripes discussed  in Refs.~\cite{Zhang18e,Liu20,Luis19}?
}


\section{Conclusions}

\bibliographystyle{splncs}
\bibliography{soliton}
\end{document}
